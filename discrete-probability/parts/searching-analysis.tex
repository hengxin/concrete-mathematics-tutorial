% file: parts/searching-analysis.tex

%%%%%%%%%%%%%%%
\begin{frame}{}
  \begin{theorem}[Computing Expectation]
    Let $X$ be a discrete random variable that takes on \red{only nonnegative integer values} $\mathbb{N}$.

    \[
      \E[X] = \textcolor{teal!60}{\sum_{i=1}^{\infty} i \Pr(X = i)} = \sum_{i=1}^{\infty} \Pr(X \ge i)
    \]
  \end{theorem}
  \pause
  \begin{proof}
    \[
      \sum_{j=1}^{\infty} \sum_{i=1}^{j} \Pr(X = j) = \sum_{i=1}^{\infty} \sum_{j=i}^{\infty} \Pr(X = j)
    \]
  \end{proof}
\end{frame}
%%%%%%%%%%%%%%%

%%%%%%%%%%%%%%%
\begin{frame}{}
  \begin{exampleblock}{Searching an Unsorted Array (CLRS Problem $5-2\; (f)$)}
    % file: algs/deterministic-search.tex

\begin{algorithm}[H]
  % \caption{Deterministic search.}
  % \label{alg:deterministic-search}
  \begin{algorithmic}[1]
    \Procedure{Deterministic-Search}{$A[1 \cdots n], x$}
      \State $i \gets 1$

      \hStatex
      \While{$i \le n$}
	\If{$A[i] = x$}
	  \State \Return \textsl{true}
	\EndIf
	\State $i \gets i + 1$
      \EndWhile

      \hStatex
	\State \Return \textsl{false}
    \EndProcedure
  \end{algorithmic}
\end{algorithm}


    \pause
    \begin{enumerate}[(a)]
      \setcounter{enumi}{4}
      \item \[ \exists!\; i: A[i] = x \]
      \item \[ \red{\exists!_{k}}\; i: A[i] = x \]
    \end{enumerate}
  \end{exampleblock}
\end{frame}
%%%%%%%%%%%%%%%

%%%%%%%%%%%%%%%
% \begin{frame}{}
%   \[ 
%     \exists!\; i: A[i] = x 
%   \]
% 
%   \[
%     \red{Y: \# \text{ of comparisons}}
%   \]
% 
%   \pause
%   \begin{align*}
%     \E[Y] &= \sum_{i = 1}^{n} i\; \red{\Pr\set{Y = i}} \\
%     \onslide<3->{&= \sum_{i = 1}^{n} i\; \red{\Pr\set{A[i] = x}} \\}
%     \onslide<4->{&= \frac{1}{n} \sum_{i = 1}^{n} i = \frac{n+1}{2}}
%   \end{align*}
% \end{frame}
%%%%%%%%%%%%%%%

%%%%%%%%%%%%%%%
\begin{frame}{}
  \[ 
    \exists!_{k}\; i: A[i] = x 
  \]

  \[
    \red{Y: \# \text{ of comparisons}}
  \]

  \pause
  \vspace{-0.50cm}
  \begin{align*}
    \E[Y] &= \sum_{i = 1}^{\red{n-k+1}} i \Pr\set{Y = i} \\
    \onslide<3->{&= \sum_{i = 1}^{n-k+1} i \Pr\set{\red{i \text{ is the first index among $k$ indices {\it s.t.} } A[i] = x}} \\}
    \onslide<4->{&= \sum_{i = 1}^{n-k+1} i \red{\frac{\binom{n-i}{k-1}}{\binom{n}{k}}}}
    \onslide<5->{= \frac{1}{\binom{n}{k}} \teal{\sum_{i = 1}^{n-k+1} i \binom{n-i}{k-1}} \onslide<6->{= \red{\cdots}} \\}
    \onslide<7->{&= \frac{1}{\binom{n}{k}} \purple{\binom{n+1}{k+1}} = \frac{n+1}{k+1}}
  \end{align*}

  \uncover<8->{
    \[
      k = 1 \implies \E[Y] = \frac{n+1}{2}, \qquad k = n \implies \E[Y] = 1
    \]
  }
\end{frame}
%%%%%%%%%%%%%%%

%%%%%%%%%%%%%%%
\begin{frame}{}
  \[
    \sum_{i=1}^{n-k+1} i \binom{n-i}{k-1} = \binom{n+1}{k+1}
  \]

  \fignocaption{width = 0.30\textwidth}{figs/qrcode-summation-wiki}
  \vspace{-0.30cm}
  \centerline{Summation by parts (\teal{Abel transformation}; wiki)}
\end{frame}
%%%%%%%%%%%%%%%

%%%%%%%%%%%%%%%
\begin{frame}{}
  \begin{exampleblock}{How Did \red{I (an ant)} Evaluate this Summation:}
    \[
      \sum_{i=1}^{n-k+1} i \binom{n-i}{k-1} = \binom{n+1}{k+1}
    \]
  \end{exampleblock}


  \begin{columns}
    \pause
    \column{0.30\textwidth}
      \fignocaption{width = 0.70\textwidth}{figs/wolfram-mma}
    \pause
    \column{0.30\textwidth}
      \fignocaption{width = 0.70\textwidth}{figs/concrete-mathematics}{\centerline{\small Chapter $5$: Binomial Coefficients}}
    \pause
    \column{0.40\textwidth}
      \[
	r \binom{r-1}{k-1} = k \binom{r}{k}
      \]

      \vspace{0.30cm}
      \[
	\sum_{0 \le k \le n} \binom{k}{m} = \binom{n+1}{m+1}
      \]
  \end{columns}
\end{frame}
%%%%%%%%%%%%%%%

%%%%%%%%%%%%%%%
\begin{frame}{}
  \begin{align*}
    \sum_{i=1}^{n-k+1} i \binom{n-i}{k-1} &= \sum_{i=0}^{n-k} (i+1) \binom{n-i-1}{k-1} \\
    &= \sum_{i=0}^{n-k} \big((n+1) - (n-i)\big) \binom{n-i-1}{k-1} \\
    &= \sum_{i=0}^{n-k} (n+1) \binom{n-i-1}{k-1} - \sum_{i=0}^{n-k} (n-i) \binom{n-i-1}{k-1} \\
    &= (n+1) \sum_{i=0}^{n-k}\binom{n-i-1}{k-1} - k \sum_{i=0}^{n-k} \binom{n-i}{k} \\
    &= (n+1) \sum_{m=k-1}^{n-1}\binom{m}{k-1} - k \sum_{m=k}^{n} \binom{m}{k} \\
    &= (n+1) \binom{n}{k} - k \binom{n+1}{k+1} = \binom{n+1}{k+1}
  \end{align*}
\end{frame}
%%%%%%%%%%%%%%%

%%%%%%%%%%%%%%%
\begin{frame}{}
  \begin{align*}
    \E[Y] &= \sum_{i=1}^{n-k+1} \red{\Pr\set{Y \ge i}} \\
    \onslide<2->{&= \sum_{i=1}^{n-k+1} \red{\frac{\binom{n-i+1}{k}}{\binom{n}{k}}} \\}
    \onslide<3->{&= \frac{1}{\binom{n}{k}} \sum_{i=1}^{n-k+1} \binom{n-i+1}{k} \\}
    \onslide<4->{&= \frac{1}{\binom{n}{k}} \sum_{r=k}^{n} \binom{r}{k} \\}
    \onslide<5->{&= \frac{1}{\binom{n}{k}} \binom{n+1}{k+1} = \frac{n+1}{k+1}}
  \end{align*}
\end{frame}
%%%%%%%%%%%%%%%
