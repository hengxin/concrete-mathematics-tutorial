% file: overview.tex

%%%%%%%%%%%%%%%
\begin{frame}{}
  \centerline{\teal{\Large Two Extra Tasks}}

  \pause
  \begin{columns}
    \column{0.50\textwidth}
      \fignocaption{width = 0.50\textwidth}{figs/chai}
    \column{0.50\textwidth}
      \pause
      \fignocaption{width = 0.80\textwidth}{figs/xiongan}
  \end{columns}

  \pause
  \vspace{0.80cm}
  \centerline{\red{\large $Q:$ What is probability?}}
\end{frame}
%%%%%%%%%%%%%%%

%%%%%%%%%%%%%%%
\begin{frame}{}
  \begin{columns}
    \column{0.35\textwidth}
      \fignocaption{width = 0.95\textwidth}{figs/infinity}
    \column{0.30\textwidth}
      \fignocaption{width = 0.70\textwidth}{figs/intuition-told}
    \column{0.35\textwidth}
      \fignocaption{width = 0.90\textwidth}{figs/dice}
  \end{columns}

  \vspace{0.50cm}
  \begin{quote}
    ``\ldots and the many \purple{paradoxes} show clearly that we, as humans, 
    \purple{lack a well grounded intuition} in this matter.''

    \hfill --- {\small ``The Art of Probability'', Richard W. Hamming}
  \end{quote}

  \pause
  \begin{quote}
    ``When called upon to judge probability, people actually judge something else
    and \purple{\bf believe} they have judged probability.''

    \hfill --- {\small ``Thinking, Fast and Slow'', Daniel Kahneman}
  \end{quote}
\end{frame}
%%%%%%%%%%%%%%%

%%%%%%%%%%%%%%%
\begin{frame}{}
  \fignocaption{width = 0.30\textwidth}{figs/Leibniz}

  \vspace{0.50cm}
  \begin{quote}
    \centerline{\red{\LARGE Let us calculate [calculemus].}}
  \end{quote}
\end{frame}
%%%%%%%%%%%%%%%
